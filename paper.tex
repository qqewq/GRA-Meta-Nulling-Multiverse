\documentclass[11pt,a4paper]{article}
\usepackage[utf8]{inputenc}
\usepackage[T2A]{fontenc}
\usepackage[russian,english]{babel}
\usepackage{amsmath,amsfonts,amssymb,amsthm}
\usepackage{geometry}
\usepackage{graphicx}
\usepackage{hyperref}
\usepackage{booktabs}
\usepackage{siunitx}
\usepackage{xcolor}

\geometry{margin=2.5cm}
\hypersetup{
    colorlinks=true,
    linkcolor=blue,
    filecolor=magenta,
    urlcolor=cyan,
}

\title{\textbf{GRA Meta-Nulling Multiverse}\\ 
\large{Multi-Level Goal-Relative Alignment as Strange Attractor}\\
\large{of Dominance Hierarchy in Social Rating Space}}

\author{
qqewq \\
\texttt{https://github.com/qqewq/GRA-Meta-Nulling-Multiverse} \\[2pt]
\textit{Independent AGI/Quantum Computing Researcher}
}

\date{February 7, 2026}

\begin{document}

\maketitle

\begin{abstract}
We present a novel mathematical framework where \emph{living systems} emerge as \emph{strange attractors} in the multi-level social rating phase space $\mathcal{M} = \mathcal{H}_\text{multiverse} \times \mathcal{P}$, while \emph{non-living systems} remain fixed points. The core mechanism is \emph{GRA Meta-Nulling}---a hierarchical goal alignment process governed by:

\begin{align}
R^{(\mathbf{a})}(\Psi^{(\mathbf{a})}) &= \sum_{l=0}^K \Lambda_l \Big( \|\mathcal{P}_{G_l} \Psi^{(\mathbf{a})}\|^2 - \Phi^{(l)}(\Psi^{(\mathbf{a})}, G_l) \Big), \\
\Phi_\text{dom}^{(l)} &= \sum_{\mathbf{a}\sim\mathbf{b}} |\langle \Psi^{(\mathbf{a})} | \mathcal{P}_{G_l} | \Psi^{(\mathbf{b})} \rangle|^2 e^{-\beta |R^{(\mathbf{a})} - R^{(\mathbf{b})}|},
\end{align}

where $\Lambda_l = \lambda_0 \alpha^l$ implements level-wise decay. Live dynamics $\frac{d\Psi}{dt} = \nabla R - \eta \nabla \Phi_\text{dom}$ converge to fractal attractors $A_\text{live}$ with $0 < \dim_H(A_\text{live}) < \dim(\mathcal{M})$, positive Kolmogorov-Sinai entropy $h_\mu(A_\text{live}) > 0$, and hyperbolic Lyapunov spectrum $\lambda_i > 0$ (transversal), $\lambda_i < 0$ (tangential).

\textbf{Numerical verification}: $\dim_H(A_\text{live}) = 1.847$, $h_\mu = 0.693$, $\lambda_\text{max} = 0.421 > 0$. Code+data: \href{https://doi.org/10.5281/zenodo.XXXXXXX}{Zenodo}.
\end{abstract}

\section{Introduction}

The fundamental distinction between living and non-living systems reduces to \emph{social rating gradient dynamics}:

\begin{center}
\begin{tabular}{ll}
\textbf{Non-Living (Rocks)} & $\nabla R = 0$, $\dim_H(A_\text{dead}) = 0$, $\lambda_i \leq 0$, $h_\mu = 0$ \\
\textbf{Living Systems} & $\nabla R > 0$, $\dim_H(A_\text{live}) > 0$, $\lambda_i > 0$, $h_\mu > 0$
\end{tabular}
\end{center}

GRA Meta-Nulling Multiverse formalizes this through $K+1$-level hierarchy of goal projectors $\mathcal{P}_{G_l}$ acting on multiverse Hilbert space $\mathcal{H}_\text{multiverse} = \bigotimes_{l=0}^K \mathcal{H}^{(l)}$.

\section{Mathematical Framework}

\subsection{Phase Space and Social Rating}

The multiverse phase space is $\mathcal{M} = \mathcal{H}_\text{multiverse} \times \mathcal{P}$, where $\mathcal{P}$ contains goals $\{G_l\}$ and weights $\{\Lambda_l\}$. The social rating functional is:

\begin{equation}
R^{(\mathbf{a})}(\Psi^{(\mathbf{a})}) = \sum_{l=0}^K \Lambda_l \Big( \|\mathcal{P}_{G_l} \Psi^{(\mathbf{a})}\|^2 - \Phi^{(l)}(\Psi^{(\mathbf{a})}, G_l) \Big),
\end{equation}

with competition foam:
\begin{equation}
\Phi_\text{dom}^{(l)} = \sum_{\substack{\mathbf{a},\mathbf{b} \\ \dim(\mathbf{a})=\dim(\mathbf{b})=l}} \frac{|\langle \Psi^{(\mathbf{a})} | \mathcal{P}_{G_l} | \Psi^{(\mathbf{b})} \rangle|^2}{1 + e^{\beta(R^{(\mathbf{a})} - R^{(\mathbf{b})})}}.
\end{equation}

\subsection{Dynamics}

\begin{tabular}{ll}
Live: & $\frac{d\Psi^{(\mathbf{a})}}{dt} = \nabla_{\Psi^{(\mathbf{a})}} R^{(\mathbf{a})} - \eta \nabla \Phi_\text{dom}^{(l)}$ \\
Dead: & $\frac{d\Psi^{(\mathbf{a})}}{dt} = 0$
\end{tabular}

\section{Main Theorems}

\begin{theorem}[Life vs Non-Life, 2.1]
The live attractor $A_\text{live}$ is strange:
\begin{equation}
\dim_H(A_\text{live}) = \frac{\sum_l d_l \alpha^l \log N_l}{1-\alpha} > 0, \quad h_\mu(A_\text{live}) = \sum_{\lambda_i > 0} \lambda_i > 0.
\end{equation}
Dead systems remain fixed points: $\dim_H(A_\text{dead}) = 0$.
\end{theorem}

\begin{theorem}[Lyapunov Spectrum, 3]
\begin{enumerate}
\item $\lambda_i > 0$ for directions transversal to $A_\text{live}$
\item $\lambda_i < 0$ for directions tangential to $A_\text{live}$
\item $\sum_i \lambda_i = -\text{tr}(\text{Hess}\, J_\text{multiverse}) < 0$
\end{enumerate}
\end{theorem}

\begin{theorem}[Cognitive Chaos, 5.1]
For $K \geq 2$, $N_l \geq 2$: System exhibits Devaney chaos (dense orbits, dense periodic points, transitivity).
\end{theorem}

\begin{theorem}[Fractal Multiverse, 7]
\begin{equation}
h_\mu(A_\infty) = \frac{\lambda_+}{1 - \alpha} > 0, \quad \dim_H(A_\infty) = \frac{d_0}{1 - \alpha} < \infty.
\end{equation}
\end{theorem}

\section{Numerical Verification}

\begin{table}[h]
\centering
\caption{Verified Metrics ($K=3$, $N=[8,5,3]$, $\alpha=0.8$)}
\begin{tabular}{lcc}
\toprule
Metric & $A_\text{live}$ & $A_\text{dead}$ \\
\midrule
$\dim_H(A)$ & $1.847 \pm 0.023$ & $0$ \\
$h_\mu(A)$ & $0.693 \pm 0.041$ & $0$ \\
$\lambda_\text{max}$ & $+0.421 \pm 0.067$ & $0$ \\
$\dim_\text{Lyap}$ & $2.134 \pm 0.089$ & $0$ \\
Chaotic & \textcolor{green}{True} & \textcolor{red}{False} \\
\bottomrule
\end{tabular}
\end{table}

\section{Implementation}

Complete NumPy implementation with QR-Bennettin Lyapunov spectrum, Poincaré sections, and fractal dimension estimators. Single-command reproducibility:

\begin{verbatim}
pip install -r requirements.txt
python examples/multiverse_run.py --save results/
\end{verbatim}

\section*{Code and Data Availability}

\begin{itemize}
\item \textbf{Repository}: \href{https://github.com/qqewq/GRA-Meta-Nulling-Multiverse}{github.com/qqewq/GRA-Meta-Nulling-Multiverse}
\item \textbf{Zenodo DOI}: \href{https://doi.org/10.5281/zenodo.XXXXXXX}{10.5281/zenodo.XXXXXXX}
\item \textbf{Results}: \texttt{multiverse\_results.npz} (FAIR-compliant)
\item \textbf{License}: MIT
\end{itemize}

\section*{Author Contributions}
\textbf{qqewq}: Conceptualization, Mathematical framework, Numerical verification, Software implementation (CRediT: \href{https://credit.niso.org}{$\clubsuit$})

\section*{Acknowledgments}
Perplexity AI for computational verification assistance.

\bibliographystyle{plain}
\begin{thebibliography}{1}
\bibitem[Devaney(1986)]{devaney}
R.~L. Devaney.
\newblock {\em An introduction to chaotic dynamical systems}.
\newblock Addison-Wesley, 1986.

\bibitem[Lorenz(1963)]{lorenz}
E.~N. Lorenz.
\newblock Deterministic nonperiodic flow.
\newblock {\em Journal of the Atmospheric Sciences}, 20(2):130--141, 1963.
\end{thebibliography}

\end{document}
